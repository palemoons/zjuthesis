\cleardoublepage

\section{绪论}

\subsection{研究背景和目的}
\par 农作物产量作为农业的重要指标,与全球饥饿问题密切相关。然而,2021年,全球范围内的多种因素叠加导致粮食价格大幅波动,进而加剧了全球粮食供应不均衡的现状,增加了粮食安全风险。新冠肺炎疫情、国际能源价格上涨、极端天气和地缘政治冲突等多重因素的影响,都使得农业生产的不稳定性逐渐加大。在这种情况下,如何及时、准确地监测农作物的生长状况、预测作物产量,成为为农业种植管理、土地利用管理、粮食贸易和食品安全决策提供科学参考的重要课题。在《中国农业产业发展报告2019》中,农业-食物系统在乡村振兴产业兴旺中扮演着至关重要的角色。该报告还强调了“口粮绝对安全,谷物基本自给”的粮食安全观在农业发展中不可或缺的地位。因此,精确监测农作物的生长状况和预测作物产量的方法和技术,对于保障粮食安全和发展农业具有非常重要的意义。然而,2022年,仍然存在许多影响农业生产的不确定因素,例如加剧的全球气候变暖、频繁发生的极端天气,这些因素都会进一步提高农业生产的不稳定性。因此,如何应对这些风险和挑战,采取有效的管理和预测措施,提高农作物的生产效率和质量,都成为当前亟待解决的重大问题。

\par 作物估产模型以及遥感影像是预测作物产量的重要手段。在目前的作物估产模型中,卷积神经网络成为了最广泛使用的深度学习算法,但是卷积神经网络在处理时序数据时存在一些问题,如无法捕捉时序数据中的长期依赖关系。为了解决这些问题,学者提出了使用长短期记忆网络解决时间序列预测问题。二者的结合固然兼顾了捕捉空间特征和处理时间序列的优点,然而,作物分布数据往往具有稀疏性,卷积计算的效率往往受到了限制,这也限制了模型在实际作物估产应用中的表现。因此,作物估产迫切需要一种新的深度学习算法,能够高效地提取遥感影像中的空间特征,同时能够捕捉时序数据中的长期依赖关系,从而提高作物估产的准确性。

\par 本研究旨在利用地学人工智能方法解决中国作物产量预测问题。通过提取遥感影像中时空信息的特征获取作物生长信息,并结合往年地区统计的作物产量数据,构建基于直方图统计的长短期记忆网络模型,提升估产精度,实现较为稳定的产量估计,为中国大面积作物估产提供方法支持和技术保障。

\section{国内外研究现状}

\subsection{研究方向及进展}

\par 近年来,随着卫星的遥感技术的进步,连续提取作物生长的时空信息成为可能。通过观察电磁波和作物之间的相互作用\cite{guan2017shared, zhang2003monitoring},遥感可以监测作物生长状况,并具有较传统方法更好的空间覆盖性、时间连续性和可用性\cite{claverie2013validation}。这种近似是可以接受的,因为归一化差异植被指数(NDVI)和增强植被指数(EVI)等植被指数已被证明对植被动态敏感\cite{sellers1987canopy, tucker1979red}并与作物产量密切相关\cite{shanahan2001use}。因此,越来越多基于遥感的产量估计方法陆续被提出,包括统计回归、传统机器学习和深度学习等方法。

\subsubsection {国内研究现状}

\par 统计回归是传统的作物估产方法。例如,Huang等人\cite{huang2013remotely}基于AVHRR NDVI数据以及相应的历史作物产量,提出了泛用于中国省级水稻产量预测的逐步回归模型。机器学习方法也被应用于作物产量估计,王鹏新等人\cite{NYJX201907026}以条件植被温度指数(VTCI)和上包络线S-G滤波的叶面积指数(LAI)为特征变量,构建了加权VTCI和LAI与玉米单产的单变量和双变量估产模型,并证明基于随机森林回归的双变量估产模型具有更好的估产效果。而随着深度学习的发展,卷积神经网络(CNN)已成为了作物估产领域最广泛使用的机器学习算法\cite{van2020crop}。例如,Yang等人设计了一个双流卷积神经网络(CNN),从无人机图像中提取产量分布的空间特征\cite{yang2019deep};Qiao等人使用空间-光谱-时间神经网络(SSTNN)对作物生长过程中的空间-光谱变化和时间依赖性进行建模\cite{qiao2021crop},并在中国的冬小麦和玉米产量预测的验证中证明了SSTNN在作物估产方面的优越性。

\subsubsection {国外研究现状}

\par 在NDVI、EVI等指数被证明与作物产量密切相关后,基于统计回归方法的作物估产方法开始被提出。例如, Shammi和Meng\cite{shammi2021use}利用基于MODIS NDVI或EVI的生长指标和密西西比州县级大豆产量建立了19个线性回归模型。Son等人\cite{son2014comparative}利用多时空MODIS EVI数据建立二次模型,于湄公河三角洲地区进行作物估产。随着机器学习的进一步发展,学者开始探索基于机器学习的作物估产方法。例如,Johnson等人\cite{johnson2016crop}以加拿大草原为研究区域,利用MODIS NDVI、EVI指数,分别构建了基于多重线性回归(MLR)、贝叶斯神经网络(BNN)和基于模型的递归分割(MOB)的作物产量预测模型,对大麦、油菜以及春小麦三种作物的产量进行预测。在近期的研究中,深度学习方法也逐步应用于作物产量预测。除了从植被指数中学习作物特征外,深度学习方法还可以进一步处理遥感图像所有光谱带中的作物生长时空特征。其中,You等人应用长短期记忆(LSTM)网络,从利用降维技术获得的MODIS直方图时间序列中挖掘时间关系,构建美国县级大豆的作物估产模型\cite{you2017deep}。

\subsubsection{当前研究问题与不足}
\par 相较于统计回归、传统机器学习等传统方法,深度学习表现出了更好的预测性能,极大程度提高了作物估产的精度。然而,由于作物分布存在稀疏性,利用卷积神经网络训练时难以利用卷积计算有效提取作物生长特征。同时,注意力机制已经被广泛应用于与序列相关的一系列问题中,通过引入该机制,模型可以自适应地关注输入序列中的重要信息,以便更好地处理各种序列数据。因此考虑从以下两方面开展研究:
\begin{itemize}
  \item [(1)] 通过构建长短期记忆神经网络,利用直方图降维技术提取作物生长特征,预测区域作物产量。同时与其他深度学习模型与传统机器学习模型进行对比,验证该模型的有效性。
  \item [(2)] 在已有长短期记忆神经网络模型中引入注意力机制进行消融实验,验证注意力机制对模型的效果。
\end{itemize}

\subsection{研究思路与技术框架}

\subsubsection{研究目的}
\par 本文的研究目标是利用结合注意力机制的长短期记忆神经网络,分别构建基于MODIS产品的东北三省地级与区县级作物总产量预测模型,并通过对比实验与传统机器学习、深度学习模型进行效果对比,验证模型有效性。

\subsubsection{研究内容}
\par 为实现上述研究目的,需从以下方面展开研究:
\begin{itemize}
  \item [(1)] 数据收集与预处理。
  \par 通过在线平台获取2000年-2012年东北三省县级玉米产量与2000年-2019年东北三省地级市玉米产量统计数据,并从开源平台获取对应区域的玉米物候数据集。同时通过NASA官方网站获取MODIS产品数据集,利用云计算平台Google Earth Engine进行数据预处理。
  \item [(2)] 估产因子提取。
  \par 通过阅读已有文献并结合自身实验,筛选能有效反应作物生长发育的遥感影像指标作为作物估产因子。
  \item [(3)] 模型构建与训练。
  \par 以中国东北地区为研究区域,选择玉米作为研究作物,以上述估产因子为输入,通过提取遥感影像中时空信息特征获取作物生长信息,并结合往年地区统计的作物产量数据,构建以结合注意力机制的长短期记忆网络为基础的地级、区县级玉米估产模型。
  \item [(4)] 估产模型评价与分析
  \par 将已有的其它模型应用于同一数据集进行产量估算,以实际产量统计数据为参考,通过计算不同模型对应的评价指标值,分析对比模型的精度和稳定性。
\end{itemize}

\subsubsection{技术路线}
\par 本文研究思路如图\autoref{fig:final_path}所示。首先通过开放平台获取了所需的玉米产量数据、作物物候分布数据集 ChinaCropPhen1km 以及 MODIS 产品数据。接下来,使用玉米物候分布数据集作为掩膜,对 MODIS 产品数据依次进行掩膜计算,以获取玉米生长期内的遥感影像指标。然后,通过运用直方图频率统计技术,对这些遥感影像指标进行降维处理,从而提取出玉米生长特征,并将其作为模型输入。结合注意力机制,构建以长短期记忆神经网络(LSTM)为基础的深度学习模型。最后,为了全面评价所提出的模型的性能表现,选择决定系数 $R^2$、均方根误差 RMSE 和皮尔森相关系数 Pearson's R 三种评价指标,与其他传统的机器学习和深度学习模型进行对比,验证模型的有效性。

\begin{figure}
  \centering
  \includegraphics[width=.8\linewidth]{final_path.png}
  \caption{研究技术路线}
  \label{fig:final_path}
\end{figure}
