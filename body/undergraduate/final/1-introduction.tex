\cleardoublepage

\section{绪论}

\subsection{研究背景和目的}
\par 农作物产量作为农业的重要指标,与全球饥饿问题密切相关。然而,2021年,全球范围内的多种因素叠加导致粮食价格大幅波动,进而加剧了全球粮食供应不均衡的现状,增加了粮食安全风险。新冠肺炎疫情、国际能源价格上涨、极端天气和地缘政治冲突等多重因素的影响,都使得农业生产的不稳定性逐渐加大。在这种情况下,如何及时、准确地监测农作物的生长状况、预测作物产量,成为为农业种植管理、土地利用管理、粮食贸易和食品安全决策提供科学参考的重要课题。在中国农业科学院发布的《中国农业产业发展报告2019》中,报告强调了“口粮绝对安全,谷物基本自给”的粮食安全观在中国农业发展中不可或缺的地位\cite{中国农业科学院2019中国农业产业发展报告}。而东北地区作为中国重要的粮食生产基地,研究该地区的作物生长特征并预测作物产量,对于保障中国粮食安全和发展国内农业具有重要意义。

\par 作物估产模型以及遥感影像是预测作物产量的重要手段。在目前的作物估产模型中,卷积神经网络(CNN)和长短期记忆神经网络(LSTM)成为了最广泛使用的深度学习算法。例如,Qiao等人通过构建空间-光谱-时间神经网络(SSTNN)使用卷积计算捕捉作物生长特征实现作物估产\cite{qiao2021crop};Sun等人则通过结合了 CNN 与 LSTM 的特性,构建了县级尺度的 CNN-LSTM 模型对美国大豆种植区进行了大豆单产预测\cite{sun2019county}。

\par 我国将卫星遥感技术应用于农业生产的时间相对较晚\cite{史舟2015农业遥感研究现状与展望}。目前,农业遥感在农业资源调查、农业灾害监测和作物产量测算等方面得到了广泛应用\cite{陈仲新2016农业遥感研究应用进展与展望}。然而,结合人工智能与大数据等新兴技术,利用遥感数据提取作物生长时空特征建立估产模型,对中国东北地区进行作物产量预测的工作尚不充足。因此,本研究旨在利用地学人工智能方法解决中国东北地区玉米产量预测精度问题。以中国东北地区为研究区,采用深度学习(Deep Learning, DL)技术,利用遥感技术获取的多光谱影像数据提取玉米生长时空特征,并结合作物真实产量数据建立深度学习估产模型,提升玉米估产精度,实现较为稳定的产量估计,为中国大面积作物估产提供方法支持和技术保障。

\section{国内外研究现状}

\subsection{研究方向及进展}

\subsubsection{作物估产方法研究进展}

\par 近年来,随着卫星的遥感技术的进步,连续提取作物生长的时空信息成为可能。通过观察电磁波和作物之间的相互作用\cite{guan2017shared, zhang2003monitoring},遥感可以监测作物生长状况,并具有较传统方法更好的空间覆盖性、时间连续性和可用性\cite{claverie2013validation}。这种近似是可以接受的,因为归一化差异植被指数(NDVI)和增强植被指数(EVI)等植被指数已被证明对植被动态敏感\cite{sellers1987canopy, tucker1979red}并与作物产量密切相关\cite{shanahan2001use}。因此,越来越多基于遥感的产量估计方法陆续被提出,包括统计回归、传统机器学习和深度学习等方法。

\par 统计回归是传统的作物估产方法。例如,Huang等人\cite{huang2013remotely}基于AVHRR NDVI数据以及相应的历史作物产量,提出了泛用于中国省级水稻产量预测的逐步回归模型。Shammi和Meng\cite{shammi2021use}利用基于MODIS NDVI或EVI的生长指标和密西西比州县级大豆产量建立了19个线性回归模型。Son等人\cite{son2014comparative}利用多时空MODIS EVI数据建立二次模型,于湄公河三角洲地区进行作物估产。

\par 统计回归方法固然简单,对数据的要求不高,能够在大多数地区得到实施,但是,简单的映射结构使其在处理作物产量和相关作物指数之间复杂的非线性关系时缺乏灵活性\cite{balaghi2008empirical, ren2008regional}。因此,随着机器学习的提出,学者开始探索基于机器学习的作物估产方法。例如,王鹏新等人\cite{NYJX201907026}以条件植被温度指数(VTCI)和上包络线S-G滤波的叶面积指数(LAI)为特征变量,构建了加权VTCI和LAI与玉米单产的单变量和双变量估产模型,并证明基于随机森林回归的双变量估产模型具有更好的估产效果。Johnson等人\cite{johnson2016crop}以加拿大草原为研究区域,利用MODIS NDVI、EVI指数,分别构建了基于多重线性回归(MLR)、贝叶斯神经网络(BNN)和基于模型的递归分割(MOB)的作物产量预测模型,对大麦、油菜以及春小麦三种作物的产量进行预测。

\par 而随着深度学习的发展,卷积神经网络(CNN)已成为了作物估产领域最广泛使用的机器学习算法\cite{van2020crop}。例如,Yang等人设计了一个双流卷积神经网络(CNN),从无人机图像中提取产量分布的空间特征\cite{yang2019deep};Qiao等人使用空间-光谱-时间神经网络(SSTNN)对作物生长过程中的空间-光谱变化和时间依赖性进行建模\cite{qiao2021crop},并在中国的冬小麦和玉米产量预测的验证中证明了SSTNN在作物估产方面的优越性。除了从植被指数中学习作物特征外,深度学习方法还可以进一步处理遥感图像所有光谱带中的作物生长时空特征。其中,You等人应用长短期记忆(LSTM)网络,从利用降维技术获得的MODIS直方图时间序列中挖掘时间关系,构建美国县级大豆的作物估产模型\cite{you2017deep}。

\subsubsection{东北地区作物估产研究现状}

\par 东北地区指我国东北的辽宁省、吉林省和黑龙江省,该地区耕地主要分布于三江平原、松嫩平原、松辽平原东北部,以及周围山前台地,是中国重要的粮食生产基地。学者针对该地区农业生产展开了多方面研究,例如Yu等人通过结合气象研究与预报(WRF)与 Noah-MP-Crop 模型参数,改善了原有模型对于大豆与玉米作物生长生物量的模拟\cite{YU2022107323}。Yao等人通过改进RS-P-YEC(Remote-Sensing-Photosynthesis-Yield estimation for Crops)模型对东北平原玉米产量进行了模拟\cite{yao2015estimation}。Li等人将环境数据与多个卫星数据结合起来,构建两种线性回归方法、三种机器学习方法和一种机器学习集成模型,实现了玉米、水稻和大豆的产量预测\cite{li2022exploring}。

\subsubsection{当前研究问题与不足}
\par 相较于统计回归、传统机器学习等传统方法,深度学习表现出了更好的预测性能,极大程度提高了作物估产的精度。然而,近年来在基于深度学习模型的大面积作物估产研究中,鲜见运用深度学习方法对中国东北地区玉米进行产量预测。同时,过去对东北地区作物产量估算的模型中,往往依赖了当地环境数据与气象数据,这也限制了模型在大范围作物估产方面的应用。

\subsection{研究思路与技术框架}

\subsubsection{研究目的}
\par 本文的研究目标是运用结合注意力机制的长短期记忆神经网络(DUAL\_ATT),分别构建基于MODIS产品的东北地区地级与区县级玉米总产量预测模型,并通过对比实验与传统机器学习、深度学习模型进行效果对比,验证模型有效性,以达到提升东北地区玉米产量预测精度的目的。

\subsubsection{研究内容}
\par 为实现上述研究目的,需从以下方面展开研究:
\begin{itemize}
  \item [(1)] 数据收集与预处理。
  \par 通过在线平台获取2000年-2012年东北三省县级玉米产量与2000年-2019年东北三省地级市玉米产量统计数据,并从开源平台获取对应区域的玉米物候数据集。同时通过云计算平台Google Earth Engine对MODIS产品数据集进行数据获取与预处理。
  \item [(2)] 模型构建与训练。
  \par 以中国东北地区为研究区域,选择玉米作为研究作物,选择中分辨率成像光谱仪(MODIS)中的地表反射率(LST)产品MOD09A1、叶面积指数(LAI)产品MOD15A2H、地表蒸发量(ET)产品MOD16A2和总初级生产力(GPP)产品MOD17A2H作为输入,通过提取遥感影像中时空信息特征获取作物生长信息,并结合往年地区统计的作物产量数据,构建以结合注意力机制的长短期记忆网络为基础的地级市、区县级玉米估产模型。
  \item [(3)] 估产模型评价与分析
  \par 将已有的其它模型应用于同一数据集进行产量估算,以实际产量统计数据为参考,通过计算不同模型对应的决定系数($R^2$)、均方根误差 RMSE 和皮尔森相关系数(Pearson's R),对比模型的作物估产精度。绘制产量预测值与真实值的坐标散点图,分析研究不同模型预测数值分布。最后,考虑测试样本的地理分布,绘制预测误差时空分布图,发掘产量误差之间潜在的时空联系。
\end{itemize}

\subsubsection{技术路线}
\par 本文研究思路如图\autoref{fig:final_path}所示。首先通过开放平台获取了所需的玉米产量数据、作物物候分布数据集 ChinaCropPhen1km 以及 MODIS 产品数据。接下来,使用玉米物候分布数据集作为掩膜,对 MODIS 产品数据依次进行掩膜计算,以获取玉米生长期内的遥感影像指标。然后,通过运用直方图频率统计技术,对这些遥感影像指标进行降维处理,从而提取出玉米生长特征,并将其作为模型输入。结合注意力机制,构建以长短期记忆神经网络(LSTM)为基础的深度学习模型。最后,为了全面评价所提出的模型的性能表现,选择决定系数($R^2$)、均方根误差(RMSE)和皮尔森相关系数(Pearson's R)三种评价指标,与其他传统的机器学习和深度学习模型进行对比;绘制产量预测值与真实值的坐标散点图,分析误差数值分布;绘制误差时空分布图,分析误差时空分布关系。

\begin{figure}
  \centering
  \includegraphics[width=.75\linewidth]{final_path.png}
  \caption{研究技术路线}
  \label{fig:final_path}
\end{figure}
