\cleardoublepage{}
\begin{center}
    \bfseries \zihao{3} 摘要
\end{center}

\par 农作物产量是农业的重要指标之一,与全国乃至世界的饥饿问题密切相关。中国是全球最大的玉米生产与消费国之一,东北地区承载了全国34.3\%的玉米生产量。因此,对中国东北地区玉米产量进行准确的提前预测,可以为我国农业种植管理、土地利用管理、粮食贸易和食品安全决策提供科学参考,对粮食安全保障具有重要意义。

\par 过去的研究表明,地表反射率、叶面积指数等遥感参量能够有效反映作物的生长特征,并已广泛应用于遥感估产中。长短期记忆神经网络等深度学习模型能够捕捉时间序列中的内在时空关系,提取并学习作物时间序列中的重要生长信息,实现作物产量的估算。然而,鲜有研究基于深度学习方法,对中国东北地区玉米作物进行多尺度产量估算。因此,研究以黑龙江省、吉林省和辽宁省三个东北省份作为研究地区,从地级市与区县两个研究尺度入手,基于直方图统计方法分别对地表反射率、叶面积指数、总蒸发量和总初级生产力数据进行降维处理,构建结合注意力机制和长短期记忆神经网络的深度学习模型DUAL\_ATT,对东北地区玉米作物进行多尺度产量估测,并从估产精度分析、预测误差分布分析和误差时空分析三个方面展开结果分析,探索了不同尺度下的玉米产量估计效果以及产量误差之间潜在的时空分布关系。

\par 研究主要结论如下:

\begin{itemize}
  \item [(1)] 在地级市尺度上,研究构建的DUAL\_ATT模型的均方根误差 RMSE 为101.21万吨,决定系数$R^2$为0.8244。县级尺度上,模型的均方根误差 RMSE 为 39.42万吨,决定系数$R^2$为0.6141。对比其他机器学习与深度学习模型,模型通过利用长短期记忆神经网络和能够更加有效地捕捉玉米生长时间序列中的内在时空关系,提取生长特征,取得了更好的估产效果。
  \item [(2)] 预测值误差分析中,DUAL\_ATT模型的误差值更加均匀分布于理想的误差值区间内,且误差值的绝对值更小,说明DUAL\_ATT模型的预测值更加接近真实值。
  \item [(3)] 误差时空分析中,高误差值普遍分布在松嫩平原地区城市和县城周边地区。分析发现,该地区属于东北地区的玉米高产量地区,且高产量样本比例较少,导致了该地区玉米产量的普遍低估现象。
\end{itemize}

\par \textbf{关键词}:作物;产量;遥感;深度学习

\cleardoublepage{}
\begin{center}
    \bfseries \zihao{3} Abstract
\end{center}

\par Crop yield is one of the important indicators of agriculture and is closely related to the problem of hunger in the country and the world. China is one of the world's largest producers and consumers of corn, and the northeastern region carries 34.3\% of the country's corn production. Therefore, accurate advance prediction of maize production in northeast China can provide scientific reference for China's agricultural plantation management, land use management, grain trade and food safety decisions, and is of great significance for food security assurance.

\par Recent studies have shown that remote sensing parameters such as surface reflectance and leaf area index can effectively reflect the growth characteristics of crops and have been widely used in remote sensing yield estimation. Deep learning models such as long and short-term memory neural networks can capture the intrinsic spatio-temporal relationships in time series, extract and learn important growth information from crop time series, and achieve crop yield estimation. However, there is still no study on multi-scale yield estimation of maize crop in northeast China based on deep learning methods. In recent years, among the studies on large-area crop yield estimation based on deep learning models, less work has been conducted on maize yield prediction for northeastern China. Therefore, the study takes three northeastern provinces, Heilongjiang, Jilin and Liaoning, as the research areas, and starts from two research scales, namely, prefecture-level cities and districts and counties, and performs dimensionality reduction on remote sensing product data based on histogram statistical methods, and constructs a deep learning model DUAL\_ATT which combines attention mechanism and long short-term memory networks to perform multi-scale yield estimation of maize crop in northeastern China, and analyzes the yield estimation accuracy, prediction error The results are analyzed from three aspects: yield estimation accuracy analysis, prediction error distribution analysis and error spatio-temporal analysis, and the potential spatio-temporal distribution relationships among the yield estimation effects and yield errors at different scales are explored.

\begin{itemize}
  \item [(1)] At the prefectural scale, the RMSE of the DUAL\_ATT model constructed in the study is 1,012,100 tons with a coefficient of determination $R^2$ of 0.8244. At the county scale, the RMSE of the model is 394,200 tons with a coefficient of determination $R^2$ of 0.6141. In comparison with other machine learning and deep learning models, the model achieves better yield estimation results by using long- and short-term memory neural network and being able to capture the intrinsic spatio-temporal relationships in the maize growth time series more effectively and extract growth features, achieving better yield estimation results.
  \item [(2)] In the prediction value error analysis, the error values of the DUAL\_ATT model are more uniformly distributed in the ideal error value interval and the absolute values of the error values are smaller, indicating that the prediction values of the DUAL\_ATT model are closer to the true values.
  \item [(3)] The high error values in the spatio-temporal analysis were generally distributed in the areas around cities and counties in the Songnen Plain region. The analysis revealed that this region belongs to the high-yielding maize areas in the northeast, and the small proportion of high-yielding samples led to the general underestimation of maize production in this region.
\end{itemize}

\par \textbf{Keywords:} crops; yield; remote sensing; deep learning