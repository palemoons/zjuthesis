\cleardoublepage{}
\begin{center}
    \bfseries \zihao{3} 摘要
\end{center}

\par 农作物产量是农业的一个重要指标,与全国乃至世界的饥饿问题密切相关。中国作为全球最大的玉米生产与消费国,能够在弱先验条件的情况下,在玉米种植早期进行大面积作物产量的准确预估尤为重要。过去的研究表明,地表反射率、叶面积指数等多光谱数据能够有效反映作物的生长特征。同时通过长短期记忆神经网络等能够捕捉时间序列中的内在时空关系的深度学习模型,可以自适应地关注输入序列中的重要生长信息,有效监测作物生长情况。然而,对于分布相对稀疏的作物图像信息,传统的卷积计算方式提取信息效率较低。因此,研究以东北省份作为研究地区,通过直方图统计的方式对遥感产品数据进行降维处理,并构建结合注意力机制和长短期记忆神经网络的深度学习模型,与其他传统机器学习和深度学习模型进行估产效果对比。主要结论如下:

\begin{itemize}
  \item [(1)] 在地级市尺度上,估产模型的均方根误差 RMSE 为101.21万吨,决定系数$R^2$为0.8244。县级尺度上,模型的均方根误差 RMSE 为 39.42万吨,决定系数$R^2$为0.6141。
  \item [(2)] 对比其他机器学习与深度学习模型,研究构建的DUAL\_ATT模型通过利用长短期记忆神经网络和能够更加有效地捕捉玉米生长时间序列中的内在时空关系,提取生长特征,取得了更好的估产效果。
\end{itemize}

\cleardoublepage{}
\begin{center}
    \bfseries \zihao{3} Abstract
\end{center}

\par Crop yield is an important indicator of agriculture and is closely related to the national and world hunger problem. As the world's largest producer and consumer of corn, it is particularly important for China to be able to make accurate predictions of crop yield over large areas early in corn planting under weak a priori conditions. Past studies have shown that multispectral data such as surface reflectance and leaf area index can effectively reflect the growth characteristics of the crop. Meanwhile, deep learning models that can capture the intrinsic spatio-temporal relationships in time series, such as LSTM neural networks, can adaptively focus on the important growth information in the input series and effectively monitor crop growth. However, for crop image information with relatively sparse distribution, the traditional convolutional computation method is less efficient in extracting information. Therefore, the study takes the northeastern province of China as the research area, and performs dimensionality reduction on remote sensing product data by means of histogram statistics, and constructs a deep learning model combining attention mechanism and LSTM to compare the yield estimation effect with other traditional machine learning and deep learning models. The main conclusions are as follows: 

\begin{itemize}
  \item [(1)] At the city-level scale, the RMSE of the estimation model is $101.21\times 10^4$ tons with the $R^2$ of 0.8244. At the county-level scale, the RMSE of the model is $39.42\times 10^4$ tons with the $R^2$ of 0.6141.
  \item [(2)] Compared with other traditional machine learning and deep learning models, the DUAL\_ATT model constructed in the study gained better yield estimation results by combining LSTM and being able to more effectively capture the intrinsic spatio-temporal relationships in the maize growth time series and extract growth features.
\end{itemize}