\section{问题提出的背景}

\subsection{背景介绍}

\par 农作物产量是农业的一个重要指标,与全国乃至世界的饥饿问题密切相关。2021年,受新冠肺炎疫情、极端天气、国际能源价格上涨以及地缘政治冲突等多重因素叠加,国际粮食价格大幅波动,造成全球粮食供应不均衡加剧,粮食安全风险陡增。2022年,诸多影响农业生产的因素仍旧存在,农业风险也仍然存在,例如加剧的全球气候变暖加剧,频发的极端天气,使得农业生产的不稳定性越来越高。《中国农业产业发展报告2019》明确提出,农业—食物系统是乡村振兴产业兴旺的“压舱石”,是保障就业的“蓄水池”,更是促进国民经济发展的“战略后院”。同时,坚持“口粮绝对安全,谷物基本自给”的粮食安全观,农作物的科学生产至关重要。因此,如何及时、有效、准确地监测农作物的生长状况、预测作物产量,来为农业种植管理、土地利用管理、粮食贸易和食品安全决策提供更加科学的参考成为了日益重要的课题。

\par 作物估产模型以及遥感影像是预测作物产量的重要手段。然而,在目前的作物估产模型中,卷积神经网络成为了最广泛使用的深度学习算法,但是卷积神经网络在处理时序数据时存在一些问题,如无法捕捉时序数据中的长期依赖关系。为了解决这些问题,学者提出了使用长短期记忆网络解决时间序列预测问题。二者结合固然兼顾了捕捉空间特征和处理时间序列的优点,但原始遥感数据先被卷积神经网络处理后,再作为时间序列进入长短期记忆网络训练,仍然存在着无法有效提取空间特征的问题。因此,作物估产迫切需要一种新的深度学习算法,能够有效地提取遥感影像中的空间特征,同时能够捕捉时序数据中的长期依赖关系,从而提高作物估产的准确性。

\subsection{本研究的意义和目的}

\par 我国作为人口大国,解决好吃饭问题始终是治国理政的头等大事,农为邦本,本固邦宁,强国必先强农,加快农业强国建设是推进中国式现代化建设的基础,在全球粮食安全问题仍然严峻的背景下,更是对世界粮食安全的贡献。因此,建立科学有效的作物估产模型能有助于国家高效统计农作物产量状况,分析中国农业产业趋势,更好地实现农业高质量发展。

\par 本研究旨在利用地学人工智能方法解决中国作物产量预测问题。通过提取遥感影像中时空信息的特征获取作物生长信息,并结合往年地区统计的作物产量数据,构建卷积长短期记忆网络模型,提升估产精度,实现较为稳定的产量估计,为中国大面积作物估产提供方法支持和技术保障。