\section{问题提出的背景}

\subsection{背景介绍}

\par 农作物产量是农业的一个重要指标,与全球饥饿问题密切相关。联合国粮食及农业组织(Food and Agriculture Organization of the United Nations)预测,全世界的人口数量将在 2050 年达到九十亿左右,伴随而来的对粮食及其他主要农产品的需求将会上升大约百分之五十。然而,粮食产量的提高面临着全球变化的严峻挑战。近些年来,在全球变暖的同时,极端温度、干旱、洪水等极端天气事件频发,严重威胁了农业生产,导致农作物产量的强烈波动。同时,据粮农组织统计,自2019年新冠疫情爆发以来世界食物不足人数在继续增加。虽然全球中度或重度粮食不安全发生率自2014年起一直在缓慢上升,但2020年的预计增幅相当于前五年的总和。近三分之一世界人口(23.7亿)在2020年无法获得充足的食物,短短一年内增加了近3.2亿人。到2030年可能仍有约6.6亿人面临饥饿,与未发生新冠疫情的情景相比增加3000万人。

\subsection{本研究的意义和目的}

\par 在全球粮食安全问题仍然严重的背景下,及时、有效、准确地监测农作物的生长状况、预测作物产量变得尤为重要,以为农业种植管理、土地利用管理、粮食贸易和食品安全决策提供科学参考。
\par 因此,本研究的目的是通过构建一种基于卷积长短期记忆网络(ConvLSTM)的作物估产模型,通过提取遥感影像中时空信息的特征获取作物生长信息,并结合往年地区统计的作物产量数据进行作物估产。