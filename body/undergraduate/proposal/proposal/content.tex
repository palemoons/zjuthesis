\section{项目的主要内容和技术路线}

\subsection{主要研究内容}

\par 本研究主要开展如下内容:
\begin{itemize}
  \item [(1)] 构建基于卷积长短期记忆网络(ConvLSTM)的作物估产模型。
  \par 以中国东北地区为研究区域,选择玉米作为研究作物,根据估产效果选取最佳指数组合,通过提取遥感影像中时空信息的特征以获取作物生长信息,并结合往年地区统计的作物产量数据,采用卷积长短期记忆网络构建县域玉米估产模型。
  \item [(2)] 构建的作物估产模型评价与分析。
  \par 将已有的其它模型应用于同一数据集进行县域产量估算,以实际产量统计数据为参考,通过计算不同模型对应的评价指标值,分析对比模型的精度和稳定性。
\end{itemize}

\subsection{技术路线}
研究以中国东北地区为研究区,首先对作物分类数据、县域玉米历年产量数据、MODIS影像数据进行预处理。对研究区中的玉米作物进行生长发育期特征分析,以及多源因子和玉米产量的相关性分析,对作物估产模型的输入变量进行选择。基于卷积长短期记忆网络构建县域玉米估产模型,使用上述预处理数据进行不同组合作为输入,评估ConvLSTM模型的精度稳定性。最后,根据评价指标值对比其他已有模型的估产效果,分析模型的优劣。

\subsection{可行性分析}
\par 首先,硬件层面……
\par 其次,已有研究验证了长短期记忆网络能从时序数据挖掘时间关系,能够提取作物生长的多时空遥感观测数据之间的相互关系,相比于循环神经网络,长短期记忆网络在序列问题方面很有效地避免了梯度消失问题。而卷积长短期记忆网络将全联接操作改为了卷积运算,能够从空间数据中有效提取特征。同时,如总初级生产力、表面温度、表面蒸发量、叶面积指数等遥感影像数据已被证明与农作物产量有密切关系。因此,基于多源数据的卷积长短期记忆网络构建县域玉米估产模型具备可行性
\par 最后,研究数据包含了2000年至2015年的县域玉米历年产量数据、作物物候和种植面积信息,以及相应地区的MODIS影像数据,影像分辨率为1千米,满足县级玉米估产模型的数据需求,影像数据可从NASA官网下载,容易获取。遥感影像数据量较大,可以使用云计算平台Google Earth Engine进行处理,实现大规模数据的并行处理,提高研究效率。