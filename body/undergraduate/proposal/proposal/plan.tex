\section{研究计划进度安排及预期目标}

\subsection{进度安排}
\par 本研究进度安排如下:
\begin{itemize}
  \item 第一阶段:2023年1月10日-2023年1月31日,确定研究方向,对已有研究数据进行预处理。
  \item 第二阶段:2023年2月1日-2023年2月28日,构建基于ConvLSTM的作物估产模型,对模型进行训练,计算评价指标。
  \item 第三阶段:2023年3月1日-2023年3月15日,优化已有模型并搭建其它对比模型,进行评价指标计算。
  \item 第四阶段:2023年3月16日-2023年4月10日,完成论文撰写。
\end{itemize}
\subsection{预期目标}
\par 本研究预期目标如下:通过构建基于卷积长短期记忆网络的作物估产模型,对玉米作物估产进行预测,使估产均方根误差(RMSE)达到0.8,决定系数($R^2$)达到0.8。能基本实现对作物产量的提前预测。