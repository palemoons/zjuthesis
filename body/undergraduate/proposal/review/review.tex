\cleardoublepage
\newrefsection
\chapter{文献综述}

\section{背景介绍}

\par 农作物产量是农业的一个重要指标,与全球饥饿问题密切相关。联合国粮食及农业组织(Food and Agriculture Organization of the United Nations)预测,全世界的人口数量将在 2050 年达到九十亿左右,伴随而来的对粮食及其他主要农产品的需求将会上升大约百分之五十。然而,粮食产量的提高面临着全球变化的严峻挑战。近些年来,在全球变暖的同时,极端温度、干旱、洪水等极端天气事件频发,严重威胁了农业生产,导致农作物产量的强烈波动。因此,大面积作物产量的早期估计对于确保粮食安全和可持续发展变得越来越重要。由于作物生长的复杂性,大面积产量的估计尤其困难。这种复杂性主要表现为作物生长具有空间异质性,进一步表现为不同地区作物的时间累积生长的巨大差异\cite{kuwata2015estimating}。大面积产量估算的关键是充分提取作物生长信息,解决作物生长的空间异质性。

\par 近年来,随着卫星的遥感技术的进步,连续提取作物生长的时空信息成为可能。通过观察电磁波和作物之间的相互作用\cite{guan2017shared, zhang2003monitoring},遥感可以监测作物生长状况,并具有较传统方法更好的空间覆盖性、时间连续性和可用性\cite{claverie2013validation}。因此,越来越多基于遥感的产量估计方法陆续被提出。

\section{国内外研究现状}

\subsection{研究方向及进展}

\par 统计回归、传统机器学习和神经网络是用以实现大面积作物估产的传统方法。受限于特征学习能力,这些方法通常使用基于卫星影像数据的植被指数(VIs)\cite{shammi2021use, stas2016comparison}或时间序列\cite{Aghighi2018Machine, nagy2018wheat}来定量描述作物生长。这种近似是可以接受的,因为归一化差异植被指数(NDVI)和增强植被指数(EVI)等植被指数已被证明对植被动态敏感\cite{sellers1987canopy, tucker1979red}并与作物产量密切相关\cite{shanahan2001use}。Shammi和Meng\cite{shammi2021use}对密西西比州基于中分辨率成像光谱仪(MODIS)NDVI或EVI的生长指标和县级大豆产量建立了19个线性回归模型。Son等人\cite{son2014comparative}通过建立二次模型,利用多时空MODIS EVI数据估计湄公河三角洲地区的产量。机器学习方法也被应用于作物产量估计,包括随机森林回归(RFR)、支持向量回归(SVR)、提升回归树\cite{abebe2022combined, aghighi2018machine}以及神经网络,如贝叶斯神经网络和人工神经网络\cite{fernandes2017sugarcane,johnson2016crop}。

\par 在近期的研究中,深度学习方法也被应用于作物产量估计。除了从植被指数中学习作物特征外,深度学习方法还可以进一步处理来自遥感图像所有光谱带中作物生长的时空特征。You等人)应用长短期记忆(LSTM)网络从MODIS直方图时间序列中挖掘时间关系\cite{you2017deep}。Yang等人设计了一个双流卷积神经网络(CNN),从无人机图像中提取产量分布的空间特征\cite{yang2019deep}。Sun等人提出了一个CNN-LSTM结构来分析作物生长的时间过程\cite{sun2019county}。Qiao等人使用递归三维卷积网络对作物生长过程中的空间-光谱变化和时间依赖性进行建模\cite{qiao2021crop}。结果显示,深度学习方法在大面积产量估计方面比其他方法表现更好。原因可能如下:首先,深度学习模型采用独特的结构进行信息处理,能够提取作物生长的多时空遥感观测数据之间的相互关系\cite{werbos1990backpropagation};其次,深度学习模型可以转换分层特征,这使得它们能够更好地揭示大面积时空特征与产量之间的复杂非线性关系\cite{lecun2015deep}。

\subsection{存在问题}

\par 统计回归方法固然简单,对数据的要求不高,能够在大多数地区得到实施,但是,简单的映射结构使其在处理作物产量和相关作物指数之间复杂的非线性关系时缺乏灵活性\cite{balaghi2008empirical, ren2008regional}。

\par 与统计方法相比,机器学习和神经网络能够更好地解决作物产量估算中的非线性问题。然而,这些方法使用的植被指数往往是几个光谱带的线性组,导致了大量光谱信息的浪费。此外,作物生长过程中的时间特征没有被充分理解,或者没有包括在现有方法中。众所周知,农作物的生长过程是一个物质积累的过程,它涉及到多时空的图像。除了从机器学习模型中提取空间和光谱特征外,还应当有适当的时间特征表示方法,以充分利用时间序列中的顺序信息。

\par 基于深度学习的作物产量估计固然有较于传统方法的优势。然而,需要进一步对深度学习模型进行研究,以阐明作物生长的空间异质性对产量估计的影响。已有研究验证了考虑作物生长的空间异质性可以有效提高大面积产量估算的准确性\cite{challinor2004design, chen2018improving, duncan2015potential, tao2009modelling},并提出了几种空间异质性学习方法,包括物候学排列、多任务学习和多模型学习。但是这些空间异质性学习方法总是严重依赖复杂的人工干预和区域划分的先验知识,如物候学划分和地理划分。而在大面积的区域划分中很难获得准确的先验知识,并且区域划分可能会切断区域间固有的相关性。此外,过多的人工干预会大大增加结果的主观性。因此,在大面积地区实现自适应空间异质性学习和自适应产量估计需要新的方法。

\section{研究展望}

\par 已有研究证明了基于深度学习的作物产量估计方法在大面积地区的实际应用中具有较好的效果。然而,这些方法仍然存在一些问题,如对时间序列的处理不够充分,对空间异质性的学习不够自适应等。因此,在未来的研究中,应当进一步探索如何从原始的多时空图像中充分挖掘时间序列关系,实现自适应空间异质性的学习,完成大面积作物产量估计。

\newpage
\begingroup
    \linespreadsingle{}
    \printbibliography[title={参考文献}]
\endgroup
