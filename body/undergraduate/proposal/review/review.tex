\cleardoublepage
\newrefsection
\chapter{文献综述}

\section{背景介绍}

\par 农作物产量是农业的一个重要指标,与全球饥饿问题密切相关。联合国粮食及农业组织(Food and Agriculture Organization of the United Nations)预测,全世界的人口数量将在 2050 年达到九十亿左右,伴随而来的对粮食及其他主要农产品的需求将会上升大约百分之五十。然而,粮食产量的提高面临着全球变化的严峻挑战。近些年来,在全球变暖的同时,极端温度、干旱、洪水等极端天气事件频发,严重威胁了农业生产,导致农作物产量的强烈波动。

\par 民以食为天,粮食安全问题在过去、现在以及未来都是国家生存与发展的重大问题。中国作为发展中国家,经济的发展更是依赖于农业经济的发展。因此,大范围可靠的农情信息对粮食市场及相关政策的制定至关重要,是保障区域及国家粮食安全的重要依据,特别是产量信息的快速、有效获取可以降低市场风险并提高效率。

\par 由于作物生长的复杂性,大面积的作物估产异常困难。这种复杂性主要表现为作物生长具有空间异质性,进一步表现为不同地区作物的时间累积生长的巨大差异\cite{kuwata2015estimating}。因此,大面积产量估算的关键是充分提取作物生长信息,解决作物生长的空间异质性。

\section{国内外研究现状}

\subsection{研究方向及进展}

\par 近年来,随着卫星的遥感技术的进步,连续提取作物生长的时空信息成为可能。通过观察电磁波和作物之间的相互作用\cite{guan2017shared, zhang2003monitoring},遥感可以监测作物生长状况,并具有较传统方法更好的空间覆盖性、时间连续性和可用性\cite{claverie2013validation}。这种近似是可以接受的,因为归一化差异植被指数(NDVI)和增强植被指数(EVI)等植被指数已被证明对植被动态敏感\cite{sellers1987canopy, tucker1979red}并与作物产量密切相关\cite{shanahan2001use}。因此,越来越多基于遥感的产量估计方法陆续被提出,包括统计回归、传统机器学习和深度学习等方法。

\subsubsection {国内研究现状}

\par 统计回归是传统的作物估产方法。例如,Huang等人\cite{huang2013remotely}基于AVHRR NDVI数据以及相应的历史作物产量,提出了泛用于中国省级水稻产量预测的逐步回归模型。机器学习方法也被应用于作物产量估计,王鹏新等人\cite{NYJX201907026}以条件植被温度指数(VTCI)和上包络线S-G滤波的叶面积指数(LAI)为特征变量,构建了加权VTCI和LAI与玉米单产的单变量和双变量估产模型,并证明基于随机森林回归的双变量估产模型具有更好的估产效果。而随着深度学习的发展,卷积神经网络(CNN)已成为了作物估产领域最广泛使用的机器学习算法\cite{van2020crop}。例如,Yang等人设计了一个双流卷积神经网络(CNN),从无人机图像中提取产量分布的空间特征\cite{yang2019deep};Qiao等人使用空间-光谱-时间神经网络(SSTNN)对作物生长过程中的空间-光谱变化和时间依赖性进行建模\cite{qiao2021crop},并在中国的冬小麦和玉米产量预测的验证中证明了SSTNN在作物估产方面的优越性。

\subsubsection {国外研究现状}

\par 在NDVI、EVI等指数被证明与作物产量密切相关后,基于统计回归方法的作物估产方法开始被提出。例如, Shammi和Meng\cite{shammi2021use}利用基于MODIS NDVI或EVI的生长指标和密西西比州县级大豆产量建立了19个线性回归模型。Son等人\cite{son2014comparative}利用多时空MODIS EVI数据建立二次模型,于湄公河三角洲地区进行作物估产。随着机器学习的进一步发展,学者开始探索基于机器学习的作物估产方法。例如,Johnson等人\cite{johnson2016crop}以加拿大草原为研究区域,利用MODIS NDVI、EVI指数,分别构建了基于多重线性回归(MLR)、贝叶斯神经网络(BNN)和基于模型的递归分割(MOB)的作物产量预测模型,对大麦、油菜以及春小麦三种作物的产量进行预测。在近期的研究中,深度学习方法也逐步应用于作物产量预测。除了从植被指数中学习作物特征外,深度学习方法还可以进一步处理遥感图像所有光谱带中的作物生长时空特征。其中,You等人应用长短期记忆(LSTM)网络,从利用降维技术获得的MODIS直方图时间序列中挖掘时间关系,构建美国县级大豆的作物估产模型\cite{you2017deep}。

\subsection{存在问题}

\par 相较于统计回归、传统机器学习等传统方法,深度学习表现出了更好的预测性能,极大程度提高了作物估产的精度,但在使用卷积神经网络与长短期记忆网络结合的训练作物估产模型时,原始遥感数据先被卷积神经网络处理后,再作为时间序列进入长短期记忆网络训练,存在着无法有效提取空间特征的问题。同时,将原始遥感数据预处理为直方图也损失了一定的位置信息。因此,现有的深度学习方法仍然无法有效提取时空信息中的空间特征,导致不能解决作物估产这一时空序列预测问题。

\section{研究展望}

\par 已有研究证明了基于深度学习的作物产量估计方法在大面积地区的实际应用中具有较好的效果。然而,这些方法仍然存在着以上所述的问题。因此,在未来的研究中,应当进一步探索如何从原始的多时空图像中充分挖掘时空序列关系,尝试直接赋予长短期记忆网络提取时空信息中特征的能力,避免空间信息的丢失,以完成时空序列预测问题。

\newpage
\begingroup
    \linespreadsingle{}
    \printbibliography[title={参考文献}]
\endgroup
